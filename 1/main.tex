\documentclass[a4paper, 12pt]{article}
\usepackage{graphicx}
\usepackage[UTF8]{ctex}
\usepackage{listings}
\usepackage{float}
\usepackage{hyperref}
\usepackage{color}

\title{实验报告1\\Git与\LaTeX}
\author{洪子翔}
\date{\today}

\lstset{
    basicstyle          =   \sffamily,
    keywordstyle        =   \bfseries,
    commentstyle        =   \rmfamily\itshape,
    stringstyle         =   \ttfamily,
    flexiblecolumns,
    numbers             =   left,
    showspaces          =   false,
    numberstyle         =   \zihao{-5}\ttfamily,
    showstringspaces    =   false,
    captionpos          =   t,
    frame               =   lrtb,   
}

\setlength{\parindent}{0pt}

\begin{document}

\maketitle
\newpage
\tableofcontents
\newpage

\section{实验内容}

    项目链接:\href{https://github.com/SoraRosa0514/system-dev-tools/tree/main/1}{https://github.com/SoraRosa0514/system-dev-tools/tree/main/1}
	\subsection{Git}
    这一部分内容使用样例项目"alice"完成。\\
    链接:\href{https://github.com/SoraRosa0514/alice}{https://github.com/SoraRosa0514/alice}
    	\subsubsection{初始配置}
            可以使用\verb|git config|实现Git的各种配置。配置变量按优先级从低到高分为三级:
            \begin{enumerate}
                \item Git路径下的/etc/gitconfig:对该系统全体用户生效。
                \item  用户路径下的~/.gitconfig:对该用户生效。
                \item 仓库中的.git/config:对该仓库生效。
            \end{enumerate}
            排在后面的config会覆盖前者。\\
            初次使用Git需要初始化用户信息,包括用户名和邮箱地址,代码如下:\\
            \verb|git config user.name "Alice"|\\
            \verb|git config user.email AliceRiddel@example.com|\\
            如果希望一劳永逸,可以在\verb|config|后加上\verb|--global|,此处的配置将作用于整个系统。
            要查看配置结果,只需要把配置的命令去掉末尾的值即可,例如:\\
            \verb|$ git config user.name|\\
            \verb|Alice|\\
            要查看所有配置内容,输入\verb|$ git config --list|
            
		\subsubsection{开始}
            可以新建仓库或克隆已有仓库。
            \begin{enumerate}
                \item 新建仓库\\
                    首先移动到对应目录,再使用\verb|git init|\\
                    成功后,应该可以在对应目录找到.git文件夹。                        
                    \begin{figure}[h]
                        \centering
                        \includegraphics[width=1\linewidth]{01.png}
                        \label{img01}
                    \end{figure}
                \item 克隆仓库 \\
                    使用\verb|git clone <repo> <directory>|\\
                    其中repo是项目地址,可以使用ssh,git,https等多种协议;directory是要保存到的本地地址。\\
                    以missing semester项目为例:\\
                        https:\\\verb|git clone https://github.com/missing-semester-cn/missing-semester-cn.github.io.git|\\
                        git:\\\verb|git clone git://...|\\
                        ssh:\\\verb|git clone git@github.com:...|
                \end{enumerate}
                
        \subsubsection{提交到github}
            首先在github初始化新仓库,之后直接按照提示输入命令即可。
            \begin{figure}[H]
                \centering
                \includegraphics[width=0.8\linewidth]{02.png}
                \label{img02}
            \end{figure}
            结果如图所示:
            \begin{figure}[H]
                \centering
                \includegraphics[width=0.8\linewidth]{03.png}
                \label{img03}
            \end{figure}
            
        \subsubsection{跟踪状态}
            可以用\verb|git status|查看仓库状态。如果在Wonderland目录下创建文件"Potion.txt",再运行这个命令,会看到如下内容:\\
            \begin{figure}[H]
                \centering
                \includegraphics[width=1\linewidth]{05.png}
                \label{img05}
            \end{figure}
            从这里我们能看出:首先我们在main分支,并且没有偏差。此外,git发现了目录下未被跟踪的"Potion.txt”。如果要提交文件,必须跟踪它。\\
            暂时不要管它。如果觉得提示太复杂,用\verb|git status -s|即可打开简略模式,效果如下\\
            \begin{figure}[H]
                \centering
                \includegraphics[width=1\linewidth]{06.png}
                \label{img06}
            \end{figure}
            前面的??代表未跟踪。已跟踪的文件的提示之后会讲解。

        \subsubsection{提交更改}
            输入\verb|git add <filename>|就可以跟踪未跟踪的文件。\\
            \begin{figure}[H]
                \centering
                \includegraphics[width=0.8\linewidth]{07.png}
                \label{img07}
            \end{figure}
            如图所示,这个命令成功让git追踪文件,并将修改放入暂存区,简略模式下由绿A标记表示新增的文件。\\
            需要注意:如果之后再对文件进行修改,需要再次暂存更改。例如在Potion.txt写下“喝下就会变大的药”,此时再检查状态:\\
            \begin{figure}[H]
                \centering
                \includegraphics[width=1\linewidth]{08.png}
                \label{img08}
            \end{figure}
            提示更改未暂存,用第二列的红M表示。再次用\verb|git add <filename>|即可。\\
            之后输入\verb|git commit -m "Test commit"|提交修改,并留下“Test commit”的留言方便阅读。\\
            为了在github上看到修改,还需要使用\verb|git push origin main|推送上去。\\

        \subsubsection{跳过暂存提交}
            如果修改了许多文件,且它们都被追踪,直接输入\\
            \verb|git commit -a|\\
            可以跳过暂存步骤。

        \subsubsection{删除文件}
            直接在工作目录删除文件是不可行的。假如本地删除Potion.txt再跟踪状态:\\
            \begin{figure}[H]
                \centering
                \includegraphics[width=1\linewidth]{09.png}
                \label{img09}
            \end{figure}
            与增加文件一样,必须将修改放到暂存区。这里需要使用\verb|git rm <filename>|才行,之后提交修改。\\

        \subsubsection{移动或重命名文件}
            只需使用\verb|git mv <file> <newfile>|\\
            例如将目录下的rabbit.txt变成cat.txt:\\
            \begin{figure}[H]
                \centering
                \includegraphics[width=1\linewidth]{010.png}
                \label{img010}
            \end{figure}

        \subsubsection{查看日志}
        输入\verb|git log|即可查看历史更改。\\
        \begin{figure}[H]
            \centering
            \includegraphics[width=1\linewidth]{011.png}
            \label{img011}
        \end{figure}
        通过日志我们能看到每次提交的作者,时间和说明等,时间更近的提交在前。\\
        要想让日志更好读,尝试\verb| git log --all --graph --decorate|\\
        \begin{figure}[H]
            \centering
            \includegraphics[width=1\linewidth]{012.png}
            \label{img012}
        \end{figure}
        这种模式会用彩色的线标注不同提交的分支关系。不过这个仓库没有。\\

        \subsubsection{查找特定提交信息}
        通过在\verb|git log|加上参数,可以方便地查找一些东西。\\
        比如说,我们尝试查看cat.txt的最后一次修改:\\
        \begin{figure}[H]
            \centering
            \includegraphics[width=1\linewidth]{013.png}
            \label{img013}
        \end{figure}
        此处使用\verb|git log -1 cat.txt|,其中"-x"代表“倒数第x次更改”。\\

    
	\subsection{\LaTeX}
 
		\subsubsection{基本文档结构}
		一个最简单的文档可以这样实现:
		\begin{lstlisting}[language=TeX] 
			\documentclass{article}
			\begin{document}
				Test test.
			\end{document}
		\end{lstlisting}
        实现的效果如图\ref{img1}。\\
        \begin{figure}[H]
            \centering
            \includegraphics[width=0.25\linewidth]{1.png}
            \label{img1}
        \end{figure}
        其中,第一行确定了文件的格式,除了article之外,还包括report,book等,也可以制定纸张类型、字体大小等参数。
        第二行和第四行分别是文件环境的开头和结尾,二者之间的内容是文档的主体。begin之前的内容是预处理命令,可以引入宏包,设置参数等;end之后的部分则会被忽略。
        
        \subsubsection{标题与目录}
            首先是标题,包括大标题,作者名和日期。
            \begin{lstlisting}[language=TeX] 
            ...
            \title{Doc 1}
            \author{Alice}
            \date{\today}
            ...
            \begin{document}
            \maketitle
            \end{document}
		  \end{lstlisting}
            实现的效果如图\ref{img2}。\\
            \begin{figure}[H]
                \centering
                 \includegraphics[width=0.4\linewidth]{2.png}
                \label{img2}
             \end{figure}
                插入标题时,title date和author指令设置标题参数,大括号内是对应的内容。设置完之后,使用maketitile指令插入标题。\\
                添加目录可以使用\verb|\tableofcontents|,可能需要多次编译。\\


        \subsubsection{文章层次}
            对article格式而言,有section,subsection和subsubsection三种层次,代码为\\
                \verb|\section{}|\\
                \verb|\subsection{}|\\
                \verb|\subsubsection{}|\\
                效果如图:\\
                \begin{figure}[H]
                    \centering
                    \includegraphics[width=0.5\linewidth]{3.png}
                \end{figure}

            
        \subsubsection{彩色字体}
            彩色字体的适配需要color宏包。\\
            让字体变色的代码是\verb|{\color{colorname}text}|\\
            改变背景色的则是\verb|\colorbox{colorname}{text}|\\
            以蓝底白色的字water为例,代码为:\\
            \verb|\colorbox{blue}{\color{white}{water}}|\\
            \colorbox{blue}{\color{white}{water}}\\

        \subsubsection{特殊字符}
            \# \$ \% \^{} \& \_ \{ \} \~{} \textbackslash \\
            上述文字是特殊字符,需要转义或特殊指令才能显示:\\
            \verb|\# \$ \% \^{} \& \_ \{ \} \~{} \textbackslash|

        \subsubsection{列表}
            分有序和无序两种。有序列表的代码为:\\
            \begin{lstlisting}[language=TeX] 
                \begin{enumerate}
                    \item A
                    \item B
                \end{enumerate}
		  \end{lstlisting}
            实现效果:\\
                \begin{enumerate}
                    \item A
                    \item B
                \end{enumerate}
            无序列表:\\
            \begin{lstlisting}[language=TeX] 
                \begin{itemize}
                    \item A
                    \item B
                \end{itemize}
		  \end{lstlisting}
            实现:
                \begin{itemize}
                    \item A
                    \item B
                \end{itemize}

        \subsubsection{表格}
        以一段代码举例:\\
        \begin{lstlisting}[language=TeX] 
            \begin{tabular}{|l|l|}
            Apples       & Green  \\
            \hline
            Strawberries & Red    \\
            \cline{2-2}
            Orange      & Orange \\
            \end{tabular}
		\end{lstlisting}
        \begin{tabular}{|l|l|}
            Apples       & Green  \\
            \hline
            Strawberries & Red    \\
            \cline{2-2}
            Orange      & Orange \\
        \end{tabular}
        \\
        第一行第二个大括号的内容是表格结构,由l,r,c和|组成。前三项每个字母对应一列,分别是左对齐、中对齐和右对齐。|是竖边框。
        中间的内容,用\&分隔每一列,双反斜线换行,\verb|\hline|代表完整横边框,\verb|\cline{x-y}|代表从x列到y列的横边框。

        \subsubsection{图表}
            要插入图表(图片)需要引入graphicx宏包,以随便的一张图为例:
            \begin{figure}[h]
                \centering
                \includegraphics[width=0.5\linewidth]{4.png}
                \caption{我是样例}
                \label{img4}
            \end{figure}
            
            \begin{lstlisting}[language=TeX] 
                \begin{figure}[h]
                    \centering
                    \includegraphics[width=0.5\linewidth]{4.png}
                    \caption{我是样例}
                    \label{img4}
                \end{figure}
		  \end{lstlisting}
            图表同样是一种环境,中间各行分别代表居中、[行宽0.5倍]{文件名4.png},标题和标签。\\
            begin后的方括号内标记了图表的放置位置,可以是h(适应),t(页首),b(页尾)等,加!代表强制。\\
            更精确的控制需要float宏包的支持。\\
            
        \subsubsection{插入代码}
        行内代码可以用\verb|\verb|||实现,例如:\\
        \verb|print("Hello tex!)|\\
        长段代码需要引入listings宏包,基本格式是:
            \begin{lstlisting}[language=TeX] 
                \begin{lstlisting}[language=C++]
                    cout << "Hello World!";
                \textbackslash end{lstlisting}
		  \end{lstlisting}
            begin后的方括号内表明代码语言,中间输入即可。\\

        \subsubsection{公式}
            行间公式可以使用\verb|\[...\]|,比如\[1+2=3\]\\
            \verb|\[1+2=3\]|
            长公式可以用equation环境。同样以1+2=3为例。\\
            \begin{equation}
                1+2=3
            \end{equation}
            要输入上下标,上标用\^{}\{...\},下标用\_\{...\}\\
            比如\[a^{up}_{down}\]
            要输入分数,用\textbackslash frac\{分子\}\{分母\}\\
            比如\[\frac{2}{5}\]
            希腊字母的输入使用\textbackslash 接希腊字母英文名称的格式,大小写由名称首字母大小写决定。\\
            更复杂的输入最好使用amsmath宏包。\\

    \section{心得}
        这节课的学习让我学到了两种非常重要的工具:便利的远程版本控制工具Git及平台GitHub,以及适合学术写作的排版系统\LaTeX。虽然相比之下这两种工具较难使用,但更加专业且便捷。可以预料到,以后我将经常和它们打交道。
\end{document}